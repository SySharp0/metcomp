%%%%%%%%%%%%%%%%%%%%%%%%%%%%%%%%%%%%%%%%%
% "ModernCV" CV and Cover Letter
% LaTeX Template
% Version 1.1 (9/12/12)
%
% This template has been downloaded from:
% http://www.LaTeXTemplates.com
%
% Original author:
% Xavier Danaux (xdanaux@gmail.com)
%
% License:
% CC BY-NC-SA 3.0 (http://creativecommons.org/licenses/by-nc-sa/3.0/)
%
% Important note:
% This template requires the moderncv.cls and .sty files to be in the same 
% directory as this .tex file. These files provide the resume style and themes 
% used for structuring the document.
%
%%%%%%%%%%%%%%%%%%%%%%%%%%%%%%%%%%%%%%%%%

%----------------------------------------------------------------------------------------
%	PACKAGES AND OTHER DOCUMENT CONFIGURATIONS
%----------------------------------------------------------------------------------------

\documentclass[11pt,letterpaper,sans]{moderncv} % Font sizes: 10, 11, or 12; paper sizes: a4paper, letterpaper, a5paper, legalpaper, executivepaper or landscape; font families: sans or roman

\moderncvstyle{casual} % CV theme - options include: 'casual' (default), 'classic', 'oldstyle' and 'banking'
\moderncvcolor{purple} % CV color - options include: 'blue' (default), 'orange', 'green', 'red', 'purple', 'grey' and 'black'

\usepackage{lipsum} % Used for inserting dummy 'Lorem ipsum' text into the template
\usepackage{}

\usepackage[scale=0.8]{geometry} % Reduce document margins
\setlength{\hintscolumnwidth}{4cm} % Uncomment to change the width of the dates column
%\setlength{\makecvtitlenamewidth}{10cm} % For the 'classic' style, uncomment to adjust the width of the space allocated to your name

%----------------------------------------------------------------------------------------
%	NAME AND CONTACT INFORMATION SECTION
%----------------------------------------------------------------------------------------

\firstname{Raimundo} % Your first name
\familyname{Ferreira} % Your last name

% All information in this block is optional, comment out any lines you don't need
\title{Curriculum Vitae}
\address{European Southern Observatory - ESO}{Karl-Schwarzschild-Straße 2, 85748 Garching bei München, Alemanha}
%\mobile{}
%\phone{}
\email{ferreirar@eso.eu}
%\homepage{facebook.com/beetmex/}{facebook.com/beetmex} % The first argument is the url for the clickable link, the second argument is the url displayed in the template - this allows special characters to be displayed such as the tilde in this example
\extrainfo{Doutorando em Astronomia}
\photo[0pt][0pt]{Resume2} % The first bracket is the picture height, the second is the thickness of the frame around the picture (0pt for no frame)

%----------------------------------------------------------------------------------------

\begin{document}

\makecvtitle {}
\subsection{Descrição geral}
\cvitem{}{Sou estudante de mestrado do ESO, na área de Astrofísica extragaláctica. Atualmente, busco desvendar a relação entre buracos negros de massa intermediária residindo no interior de aglomerados globulares, e quais as relações dessas estruturas com a formação da galáxia (analisando parâmetros gerais da mesma), e também com a matéria escura. Por fim, sou integrante da equipe de comissão da IAU.}

%----------------------------------------------------------------------------------------
%	EDUCATION SECTION
%----------------------------------------------------------------------------------------

\section{Formação acadêmica/titulação}
\cventry{2026 - atual}{Mestrado em Astrofísica}{Universidade de Cardiff}{Reino Unido}{\textit{Estudo das propriedades de galáxias anãs e seu reflexo na população de aglomerados globulares (com buracos negros supermassivos em seu interior), um marcador de matéria escura e processos evolutivos}}{}

\cventry{2022 - 2025}{Graduação em Astronomia}{Universidade Federal do Rio de Janeiro}{}{\textit{Observatório do Valongo}}{}

\cventry{2019 - 2021}{Ensino médio (2º grau)}{}{}{\textit{Colégio Social de Jequié}}{}  % Arguments not required can be left empty


\section{Conquistas}
\cvitem{2025}{Graduação finalizada com o grau “Cum Laude” (Média acadêmica > 8.0)}
\cvitem{2024}{Aprovação para estagiar na escola de verão do ESO}
\cvitem{2023}{Duas menções honrosas na Semana de Integração Acadêmica da UFRJ - 2023}
\cvitem{2021 - 2020}{Bi-medalhista de ouro na Olimpíada Brasileira de Astronomia.}
\cvitem{2021}{Aprovação em 2º lugar da Bahia na 1ª fase da Olimpíada Brasileira de Satélites com direito ao kit OBSAT para montagem de CUBESAT, Ministério de Ciências Tecnologia e Informação.}
%----------------------------------------------------------------------------------------
%	WORK EXPERIENCE SECTION
%----------------------------------------------------------------------------------------
\section{Atuação profissional}

\cventry{2026 - atual}{Projeto de mestrado}{\textsc{Estudo de aglomerados globulares populados por buracos negros de massa intermediária em galáxias anãs}}{Orientador: Katharina Immer}{Colaborador: Marco Grossi, Duncan A. Forbes, S.S. Larsen}{
Entender como buracos negros de massa intermediaria se formaram e evoluíram dentro de estruturas fundamentais para traçar diversos parâmetros da evolução do Universo. Desvendar as relações de ambas estruturas matriz, atrelados a estruturas fundamentais para a evolução do Universo como as galáxias anãs, fazem esse estudo ser essencial para resolver diversos questionamentos da Astrofísica Extragaláctica e da Cosmologia.
\newline{}
}
\cventry{2024}{Estágio de Verão}{\textsc{Star formation in the extreme}}{Orientador: Ashley Barnes e Katharina Immer}{}{
One of the major unanswered questions in astronomy is: how do stars form, and have they always formed in the same way? Investigating this question has pushed us to the most extreme and violent environments, which are similar to those at much earlier times in the universe. In this project, we invite you to join us in the study of star formation within the closest of such an extreme environment - the centre of the Milky Way.
\newline{}
}

\cventry{2022 - atual}{Iniciação científica}{\textsc{Estudo das propriedades físicas e químicas de aglomerados globulares em galáxias anãs}}{Orientador: Marco Grossi}{Colaboradores: Arianna Cortesi, Denise Gonçalves, Natanael Oliveira}{
Aglomerados globulares revelam os principais eventos de formação estelar de uma galáxia, sendo assim, são representações das condições físicas que levaram a formação da mesma. Existem para galáxias massivas uma intrínseca relação entre o número de aglomerados ($N_{GC}$) e a massa do halo($M_{halo}$), entretanto ainda não há estudos que comprovem essa mesma relação nas galáxias de baixa massa. Dessa forma, buscamos determinar para galáxias anãs do grupo local suas relações $N_{GC} - M_{halo}$, da mesma forma que determinaremos as propriedades físicas dos mesmos aglomerados, como cor, distribuição espacial e estimativas de metalicidade e idade através de fotometria no ótico e no IR médio integrados. 
\newline{}
}
%------------------------------------------------


%------------------------------------------------

%----------------------------------------------------------------------------------------
%	AWARDS SECTION
%----------------------------------------------------------------------------------------





%----------------------------------------------------------------------------------------
%	COMPUTER SKILLS SECTION
%----------------------------------------------------------------------------------------

\section{Habilidades de computação}

\cvitem{Básico}{\textsc{CSS, HTML, JavaScript}}
\cvitem{Intermediário}{\textsc{ds9, Topcat, C\#, Git, GitHub} Unity, R}
\cvitem{Avançado}{\textsc{Python (Astropy, Matplotlib, Pandas), SExtractor, Linux, IRAF, \LaTeX}}

%----------------------------------------------------------------------------------------
%	COMMUNICATION SKILLS SECTION
%----------------------------------------------------------------------------------------

\section{Produções}
\subsection{Trabalhos completos publicados em anais de congressos}
\cvitem{2022}{\small{FERREIRA FILHO, R. J.; MELLO, D. ; MELLO, A. B. ; NADER, R. V. ; ABREU, E. F. ; LAVERSVEILER, M. A. ; SILVA, J. A. ; FONSECA, S. L. ; MARINHO, L. ; PACIFICO, M. ; SALCIDES, P.} \textsc{Proceeding - The return of Programa de Visitação Pública ao Observatório do Valongo/UFRJ.} XLV Reunião Anual da SAB}

\subsection{Apresentações de Trabalho}
\cvitem{2023}{\small{FERREIRA FILHO, R. J.; GROSSI, M.} \textsc{Apresentação oral - Estudo das propriedades físicas e químicas de aglomerados globulares em galáxias anãs} Universidade de Cardiff - Congresso internacional sobre galáxias anãs}
\cvitem{2023}{\small{FERREIRA FILHO, R. J.; GROSSI, M.} \textsc{Apresentação oral - Estudo das propriedades físicas e químicas de aglomerados globulares em galáxias anãs} Semana de Integração Acadêmica}
\cvitem{2023}{\small{FERREIRA FILHO, R. J.; MELLO, D. ; MELLO, A. B. ; NADER, R. V. ; ABREU, E. F. ; LAVERSVEILER, M. A. ; SILVA, J. A. ; FONSECA, S. L. ; MARINHO, L. ; PACIFICO, M. ; SALCIDES, P.} \textsc{Apresentação oral - Astronomia no Morro: O programa de visitação pública do Observatório do Valongo} Semana de Integração Acadêmica}
\cvitem{2022}{\small{FERREIRA FILHO, R. J.; MELLO, D. ; MELLO, A. B. ; NADER, R. V. ; ABREU, E. F. ; LAVERSVEILER, M. A. ; SILVA, J. A. ; FONSECA, S. L. ; MARINHO, L. ; PACIFICO, M. ; SALCIDES, P.} \textsc{Apresentação de pôster - The return of Programa de Visitação Pública ao Observatório do Valongo/UFRJ.} XLV Reunião Anual da SAB}

\subsection{Organização de eventos}

\cvitem{2022 - atual}{\small{FERREIRA FILHO, R. J.; MENENDEZ-DELMESTRE, K. ; GONCALVES, D. R. ; GROSSI, M. ; OLIVEIRA, N. G. ; SILVA, L. R} \textsc{Comissão de organização dos seminários do Observatório do Valongo}}

\subsection{Artigos publicados}
\cvitem{2023}{\small{FERREIRA FILHO, R. J.; GROSSI, M.} \textsc{Unraveling the physical and chemical properties of globular clusters in dwarf galaxies} MNRAS}

%----------------------------------------------------------------------------------------
%	LANGUAGES SECTION
%----------------------------------------------------------------------------------------

\section{Idiomas}

\cvitemwithcomment{Português}{Língua mãe}{}
\cvitemwithcomment{Inglês}{Intermediário}{Compreende bem, Fala bem, Lê bem, Escreve bem.}
\cvitemwithcomment{Espanhol}{Básico}{Compreende bem, Fala bem, Lê bem, Escreve bem.}

%----------------------------------------------------------------------------------------
%	INTERESTS SECTION
%----------------------------------------------------------------------------------------

\section{Eventos}
\cvitem{2025}{Reunião geral da IAU}
\cvitem{2023}{First Rio Dark Matter Meeting.}
\cvitem{2025}{Programa de verão do ESO}
\cvitem{2023}{Semana de Integração Acadêmica (SIAc).}
\cvitem{2022}{Workshop de Astroquímica Experimental do Observatório do Valongo/UFRJ.}
\cvitem{2022}{XLV Reunião Anual da Sociedade Astronômica Brasileira. O Retorno do Programa de Visitação Pública do Observatório do Valongo.}
\cvitem{2021}{1º Seminário Internacional: Astronomia e Astronáutica - MCTI.}
\cvitem{2021}{Seletiva para as Olimpíadas Internacionais de Astrofísica de 2021. }
% 

%----------------------------------------------------------------------------------------
%	REFERENCE SECTION
%----------------------------------------------------------------------------------------

\section{Outros}

\cvitem{}{\textsc{Participou de observações astronômicas no Observatório Pico dos Dias (OPD) durante 10 noites, auxiliando diretamente no controle de qualidade das imagens por meio de análises com o software IRAF.
Observou também no SOAR Telescope, durante 5 noites, e no JCMT por 3 noites.}}


\end{document}
